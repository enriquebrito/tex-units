% !TeX spellcheck = en_GB

\providecommand{\TITLE}{Unidades de longitud en  \TeX}

\providecommand{\INTRO}{%
   Estas conversiones pueden tener errores de redondeo y son sólo una guía.
   
   Las reglas mostradas son de \SI{1}{\mm} de resolución. Por favor, tenga en cuenta que las reglas muy cortas pueden no ser representadas correctamente en pantalla (o impresora) y parecer más largas de lo que son.
   %\linebreak donde se necesite
}

\providecommand{\ABSOLUTEUNITSHEADLINE}{Unidades absolutas}

\providecommand{\ABSOLUTEUNITSCONTENT}{%
   \TestUnit
      {Punto escalado}
      {El punto escalado se define como  1/\num{65536} puntos.}
      [Esta es la unidad más pequeña que \TeX\ usa.]
      {sp}
   \TestUnit{Punto}{El punto se define como 1/\num{72.27} de pulgada.}{pt}
   \TestUnit
      {Punto grande (DTP o punto PostScript)}
      {El punto grande se define como 1/\num{72} de pulgada.}
      [Word, InDesign y otras aplicaciones DTP usan esta definición para puntos.]
      {bp}
   \TestUnit{Punto Didot}{Una unidad antigua usada por impresores europeos}{dd}
   \TestUnit{Milímetro}{Unidad SI}{mm}
   \TestUnit{Pica}{Una pica equivale a doce puntos.}{pc}
   \TestUnit{Cícero}{Un Cícero equivale a doce puntos Didot.}{cc}
   \TestUnit{Centímetro}{Unidad SI}{cm}
   \TestUnit{Pulgada}{Una pulgada equivale a \num{2.54} centímetros.}{in}
}

\providecommand{\RELATIVEUNITSHEADLINE}{Unidades relativas}

\providecommand{\RELATIVEUNITSPRETEXT}{%
   Estas unidades dependen del tamaño de fuente activo en el momento.
   
   Para más detalles sobre em y ex consultar:
   \url{http://tex.stackexchange.com/q/4239/4918}
}

\providecommand{\RELATIVEUNITSCONTENT}{%
   \TestUnit*
      {em}
      {Tradicionalmente, un em se define como el ancho de la M mayúscula o igual al tamaño de fuente, pero en la actualidad el valor se define en el fichero de fuente.
      }
      [Esta unidad debe usarse para todas las distancias horizontales que deben cambiar en relación al tamaño de fuente; el sangrado de párrafo, por ejemplo.]
      {em}
   \TestUnit*
      {ex}
      {Tradicionalmente, un ex se define como el alto de la x minúscula, pero en la actualidad el valor se define en el fichero de fuente.}
      [Esta unidad debe usarse para todas las distancias verticales que deben cambiar en relación al tamaño de fuente.]
      {ex}
   \TestUnit*
      {mu}
      {Esta unidad equivale aproximadamente a 1/18 de em de la familia de fuente matemática.}
      [Solamente puede usarse para espaciar en modo matemático.]
      {mu}
}

\providecommand{\INFOTEXT}{%
   Copyright \raisebox{-0.2ex}{©} 2016, Tobias Weh (\href{http://tobiw.de/en}{tobiw.de/en})\\
   Source code available at: \url{https://github.com/tweh/tex-units}\\
   Versión en español de Enrique Brito
}